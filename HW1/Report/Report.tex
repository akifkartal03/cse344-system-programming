\documentclass{article}
\usepackage[utf8]{inputenc}
\usepackage[margin=1in,includefoot]{geometry}

% Header and Footer Setup
\usepackage{fancyhdr}
\pagestyle{fancy}
\fancyhead{}
\fancyfoot{}
\fancyfoot[R]{\thepage}
\renewcommand{\headrulewidth}{0pt}
\renewcommand{\footrulewidth}{0pt}
%
%Graphics Setup
\usepackage{graphicx}
\usepackage{float}
\usepackage{subfig}

%list setup
\usepackage{amssymb}
\renewcommand{\labelitemi}{$\blacktriangleright$}
\renewcommand{\labelitemii}{$\bullet$}
\renewcommand{\labelitemiii}{$\circ$}

%Source Code setup
\usepackage{xcolor}
\usepackage{listings}

\definecolor{mGreen}{rgb}{0,0.6,0}
\definecolor{mGray}{rgb}{0.5,0.5,0.5}
\definecolor{mPurple}{rgb}{0.58,0,0.82}
\definecolor{backgroundColour}{rgb}{0.95,0.95,0.92}

\lstdefinestyle{CStyle}{
    backgroundcolor=\color{backgroundColour},   
    commentstyle=\color{mGreen},
    keywordstyle=\color{magenta},
    numberstyle=\tiny\color{mGray},
    stringstyle=\color{mPurple},
    basicstyle=\footnotesize,
    breakatwhitespace=false,         
    breaklines=true,                 
    captionpos=b,                    
    keepspaces=true,                 
    numbers=left,                    
    numbersep=5pt,                  
    showspaces=false,                
    showstringspaces=false,
    showtabs=false,                  
    tabsize=2,
    language=C
}
%


\begin{document}

\begin{titlepage}

	\begin{flushright}
	\textsc{\large March 24, 2021} \\
	\end{flushright}
	\begin{center}
	\Large{\bfseries GTU Department of Computer Engineering \\ CSE344 - Spring 2021 \\ Homework 1 Report  } \\
	\end{center}
	\topskip0pt
	\vspace*{\fill}
	\begin{center}
	\Large{\bfseries Akif Kartal \\ 171044098 }
	\end{center}
	\vspace*{\fill}

\end{titlepage}

\cleardoublepage
\section{Problem Definition}
The problem is to write an "advanced" file search program for POSIX compatible operating systems. 

\section{Solution}
In order to write this program we need to \textbf{divide problem into modules}. These modules are following;

\subsection{Argument Handling}
In order to manage arguments in a proper way I used \textbf{argument struct} to keep given argument data as one block data. My argument struct is following.
\begin{lstlisting}[style=CStyle]
typedef struct st
{
    int wFlag;
    int fFlag;
    int bFlag;
    int tFlag;
    int pFlag;
    int lFlag;
    char *wArg;
    char *fArg;
    char *bArg;
    char *tArg;
    char *pArg;
    char *lArg;
    int isFound;
    int count;

} args;
\end{lstlisting}
In this struct \textbf{count} represent number of optional arguments and \textbf{isFound} represent file is found or not.
Also, in order to get argument's value \textbf{getopt()} library method was used.
\subsection{Error Handling}
In order to handle any error \textbf{stderr} used with \textbf{write system call} and 
\textbf{exit} system call was used.
Following code shows an example of this;
\begin{lstlisting}[style=CStyle]
void my_fprintf_with_stderr(const char *str){
    ssize_t size = strlen(str);
    if (size != write(STDERR_FILENO, str, size)) {
        perror("write system call error!");
        exit(EXIT_FAILURE);
    }

}
\end{lstlisting}
\textbf{Usage:}
\begin{lstlisting}[style=CStyle]
if (stat(path, &fileStat) == -1)
{
    char *str = "Stat system call error!!\n";
    my_fprintf_with_stderr(str);
    exit(EXIT_FAILURE);
}   
\end{lstlisting}
\subsection{Advanced Search Algorithm}
\subsubsection{Regex Handling}
The filename argument can contain more than one \textbf{"+"} character. To handle this situation in an easy way somehow we need to keep position data of these regexs. In order to do this I created my own \textbf{LinkedList} data structure to keep both position and previous letter information. \textbf{My LinkedList} node is following;
\begin{lstlisting}[style=CStyle]
//regex information node
typedef struct node_s
{
    int position;
    char preChr;
    struct node_s *next;

} node_t;
\end{lstlisting}
\subsubsection{Checking Given Parameters}
\begin{itemize}
	\item File name check was made by using regex linkedlist.
	\item File size, file type, file permissions and file links was checked by using \textbf{stat calls}. 
\end{itemize}
\subsubsection{Recursive Search Algorithm}
TBA.

\subsubsection{Drawing nicely formatted tree}
If the searching file is found then directory tree will be drawn by using same recursive search algorithm with just a \textbf{minor} difference. 
\subsection{CTRL-C Handling}
In order to give a message on CTRL-C interrupt, I used \textbf{signal} function from \textbf{signal.h} library.

\end{document}