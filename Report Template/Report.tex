\documentclass{article}
\usepackage[utf8]{inputenc}
\usepackage{lipsum}
\usepackage[margin=1in,includefoot]{geometry}

% Header and Footer Setup
\usepackage{fancyhdr}
\pagestyle{fancy}
\fancyhead{}
\fancyfoot{}
\fancyfoot[R]{\thepage}
\renewcommand{\headrulewidth}{0pt}
\renewcommand{\footrulewidth}{0pt}
%
%Graphics Setup
\usepackage{graphicx}
\usepackage{float}
\usepackage{subfig}


\begin{document}

\begin{titlepage}

	\begin{flushright}
	\textsc{\large March 12, 2021} \\
	\end{flushright}
	\begin{center}
	\Large{\bfseries GTU Department of Computer Engineering \\ CSE344 - Spring 2021 \\ Homework X Report  } \\
	\end{center}
	\topskip0pt
	\vspace*{\fill}
	\begin{center}
	\Large{\bfseries Akif Kartal \\ 171044098 }
	\end{center}
	\vspace*{\fill}

\end{titlepage}

\cleardoublepage
\section{Problem Definition}
The problem is to find if a subset of given array elements can sum up to the target num. 

This problem called as SUBSETSUM which is one of the popular problem in dynamic
programming.
\textbf{Inputs:} arr(Array), num(target sum), size(size of Array) \\
\textbf{Output:} Possible or Not possible (0 or 1)

\section{Solution}
Since, this problem is a recursive backtracking problem I made my algorithm by using
following approach. \\

\subsection{Functions}
First function is given function in homework I used it to check some extreme cases such as
num is equal to 0 which is already Not Possible case and to initialize variables such as total
sum. \\ \\

	
\end{document}